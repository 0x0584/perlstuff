% Created 2018-01-19 Fri 16:22
% Intended LaTeX compiler: pdflatex
\documentclass[captions=tableheading]{article}
\usepackage[utf8]{inputenc}
\usepackage[T1]{fontenc}
\usepackage{graphicx}
\usepackage{grffile}
\usepackage{longtable}
\usepackage{wrapfig}
\usepackage{rotating}
\usepackage[normalem]{ulem}
\usepackage{amsmath}
\usepackage{textcomp}
\usepackage{amssymb}
\usepackage{capt-of}
\usepackage{hyperref}
\usepackage[margin=0.85in]{geometry}
\hypersetup {
colorlinks,
citecolor=black,
filecolor=black,
linkcolor=blue,
urlcolor=blue
}
\date{\today}
\title{Notes: Perl Programming Language}
\hypersetup{
 pdfauthor={Anas Rchid},
 pdftitle={Notes: Perl Programming Language},
 pdfkeywords={},
 pdfsubject={},
 pdfcreator={Emacs 25.3.1 (Org mode 9.1.5)}, 
 pdflang={English}}
\begin{document}

\maketitle
\tableofcontents

\section{Intruduction}
\label{sec:orgd3df6ce}
hello, this is Perl!

\begin{verbatim}
#!/usr/bin/perl
# run: chmod +ox
print @ARGV;
\end{verbatim}
\captionof{figure}{code \#0}

\subsection{data types}
\label{sec:org85bb76b}
\begin{quote}
\$scalar --- "A scalar is a value that Perl treats as a single unit, like a number or a word"
\end{quote}

\begin{quote}
@array --- "An array is an ordered collection of elements, where the elements are scalars"
\end{quote}

\subsection{modules}
\label{sec:org08f3623}
\subsubsection{pragma}
\label{sec:org8d3c8c1}
\begin{quote}
A pragma is a module that controls the behavior of Perl. 
\end{quote}

\begin{quote}
strict pragma change the behavior of the perl compiler. \\
Use the pragma using \texttt{use strict;} --- pre-pend it to your source code.
\end{quote}

\subsection{subroutines}
\label{sec:org0710fc6}
\begin{verbatim}
# this is the same as code #0, it's just more complicated
sub echo {
  print "@_\n";
}
echo @ARGV
\end{verbatim}
\captionof{figure}{code \#1}

\begin{itemize}
\item All subroutine declarations start with sub followed by the name of the subrou- tine and the body.
\item The body of the subroutine is a block of statements enclosed in squiggly-braces.
\end{itemize}

\begin{quote}
In this case, the block contains a single statement.
\end{quote}
\end{document}